\documentclass[10pt,a4paper]{book}
\usepackage[spanish]{babel}
\usepackage[utf8]{inputenc}
\usepackage{amsmath}
\usepackage{amsfonts}
\usepackage{amssymb}

\usepackage{xcolor}


\usepackage[margin=0.9in]{geometry}



\title{Apuntes Stats}
\author{Oskar Denis Siodmok}
\begin{document}
\maketitle

\setcounter{chapter}{1}
\chapter*{\color{blue}\textbf{1} \color{black} Introducción}


\section{Conceptos Básicos}
\begin{itemize}
	\item Individuos o elementos: Contienen la información a estudiar.
	\item Población: Conjunto de individuos o elementos que presentan la variable a estudiar.
	\item Muestra: Subconjunto representativo de una población.
	\item Variables: Propiedades de los elementos de la población que tendrán sus respectivos valores
	\item Clases: Conjunto de valores que cumplen una propiedad. Por definición un valor solo puede pertenecer a una clase (por ejemplo, intervalos en variables contínuas).
	\item Parámetro: Será una función que operará con las diferentes variables y valores sobre la población con un propósito.
	\item Estadístico: Será una aproximación al parámetro a partir de una muestra.
\end{itemize}

\subsection{Variables Estadísticas}
Las variables estadísticas, ya definidas en el anterior apartado, se denotarán mediante una letra mayúscula (por lo general con \(X\) o \(Y\)). Podrán tomar cualquier valor de cualquier conjunto. El dominio de la variable será el conjunto de todos los posibles valores de dicha variable.

\subsection{Tipos de variables}
\begin{itemize}
	\item Variables Cuantitativas: Se expresan en cantidades numéricas o cualquier otro sistema que se pueda ordenar. A su vez se clasifican en:
		\begin{itemize}
			\item Variables Discretas: Toman valores concretos de conjuntos finitos o infinitos.
			\item Variables Continuas: Toman valores de conjuntos infinitos y no concretos (el dominio son valores continuos, como todos los reales en un intervalo).
		\end{itemize}
	Para muchas variables resulta complicado distinguir el tipo. Por ejemplo, aunque la altura matemáticamente sea continua, en la vida real nadie va a determinar una altura por encima de 2 decimales.
	\item Variables Cualitativas: No se pueden medir, solo clasificar. Un tipo concreto de este tipo de variables son las \textit{Variables Ordinales}, las cuales pese a no tener un valor numérico si que pueden tener relaciones de orden.  
\end{itemize}

\subsection{Representación de datos}
Esta se puede realizar de varias formas: 
\begin{itemize}
	\item Tablas y Gráficos: representan información de forma rápida y visual.
	\item Medidas Descriptivas: describen la información de forma numérica.
\end{itemize}

\section{Tablas de Frecuencias}
Se pueden realizar sobre cualquier conjunto de datos y sobre cualquier variable. Por ejemplo, dada una población de \(n\) indivíduos que presenta la variable \(X\) se obtienen las clases \(c=\{c_1,c_2,\dots,c_k\}\) posibles. En este caso, \(n_i\) hará referencia al número de observaciones para \(i\in\{1,\dots,k\}\subseteq\mathbb{N}\). De esta forma, \(n = \sum_in_i\) será el número total de observaciones de nuestra variable, independientemente de la clase de cada observación. Por otro lado, la frecuencia relativa de una variable representará la frecuencia de una variable (\(n_i\)) sobre \(1\) y se calculará como $\frac{n_i}{n}$.
\begin{center}
\begin{tabular}{r|c|c|c|c|l}
	Clase \(c_i\) & \(c_1\) & \(c_2\) & $\dots$ & \(c_k\) & \\
	\hline
	Freq. Absoluta \(n_i\) & $n_1$ & $n_2$ & $\dots$ & $n_k$ & $n=\sum_in_i$\\
	\hline
	Freq. Relativa $f_i$ & $f_1$ & $f_2$ & $\dots$ & $f_k$ & $f = \sum_if_ii$\\
\end{tabular}
\end{center}

Se podrían añadir las frecuencias acumuladas, las cuales se van sumando a los datos anteriores. Siendo $N_i$ la frecuencia absoluta acoumulada y $F_i$ la frecuencia relativa acumulada: 
\begin{center}
\begin{tabular}{r|c|c|c|c|l}
	Clase \(c_i\) & \(c_1\) & \(c_2\) & $\dots$ & \(c_k\) & \\
	\hline
	Freq. Absoluta \(n_i\) & $n_1$ & $n_2$ & $\dots$ & $n_k$ & $n=\sum_in_i$\\
	\hline
	Freq. Relativa $f_i$ & $f_1$ & $f_2$ & $\dots$ & $f_k$ & $f = 1$\\
	\hline
	F. abs. acum. $N_i= \sum\limits_{j=1}^in_j$ & $N_1$ & $N_2$ & $\dots$ & $N_k$ & $N = N_k = n$\\
	\hline
	F. rel. acum $F_i= \sum\limits_{j=1}^if_j$ & $F_1$ & $F_2$ & $\dots$ & $F_k$ & $F = F_k = 1$\\

\end{tabular}
\end{center}

Si la variable es cualitativa entonces las clases serán nominales. Si la variable es discretas las clases serán valores numéricos dentro del rango y si es continua serán intervalos $(l_{i-1},l_i]\:\forall i$. Si las clases son intervalos habrá varios parámetros y conceptos de interes: 
\begin{itemize}
	\item Amplitud del intervalo: $a_i = l_i -l_{i-1}$.
	\item Marca de clase: Será el valor representativo del intervalo. Por ejemplo, el punto medio: $c_i = \frac{l_i + l_{i-1}}{2}$. Podrá representarse en la segunda columna o fila de la tabla.
	\item Número de intervalos: Se realizará mediante aproximación pues se pueden plantear los intervalos que se quiera. Una elección típica es: 
		\[k = \begin{cases}
			\sqrt{n} & ,n \text{ no es muy grande}\\
			1+\log_2(n) 
		\end{cases}\]
	\item Normalización: Cuando el intervalo tiene muchos decimales conviene redondear hacía arriba para que los datos sean más legibles. Notar que si se redondea a la baja se altera el número de intervalos, por lo cual no conviene. 
	\item Límites de los intervalos: Por definición de clase, un valor de una variable solo puede pertenecer a una clase. Debido a ello hay que tener en cuenta los límites de los intervalos prestando atención a que ningún par de clases ccontenga valores repetidos.
\end{itemize}
\section{Gráficos}
\subsection{Diagrama de Barras}
En el eje $X$ se representan las clases y en el $Y$ las frecuencias absolutas o relativas (ya que son proporcionales la escala se mantendría). Busca en google como son que no tengo ningún conjunto de datos interesante. Si el diagrama de barras se hace a partir de una variable continua y sus intervalos entonces será un histograma. Ante intervalos de mimas amplitud, el gráfico resultante será proporcional a su correspondiente historgrama de frecuencias relativas, esta característica se tendrá que cumplir siempre. Otra vez, hay que tener cuidado con el número de intervalos seleccionados para que el histograma muestre la información de forma óptima. Según la forma del histograma puede ser:
\begin{itemize}
	\item Distribución unimodal simétrica: Los datos tienen forma de campana de gauss.
	\item Distribución bimodal simétrica: Hay cierta simetría respecto al eje que divide los datos por la mitad pero no es necesariamente gaussiana. 
	\item Distribución asimétrica a la derecha: Hay escasez de datos a la derecha.
	\item Distribución asimétrica a la izquierda: Hay escasez de datos a la izquierda. 
\end{itemize}
\subsection{Diagrama de Sectores}
Se divide un círculo en sectores proporcionales a las frecuencias. El ángulo de cada clase se calcularía con una simple regla de tres:
\[\frac{n}{n_i} = \frac{2\pi}{x_i}\]

\end{document}



\end{itemize}

